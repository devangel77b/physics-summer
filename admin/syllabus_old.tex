\documentclass{article}

\title{Physics Summer Syllabus}
\author{Dennis Evangelista}
\date{\today}

\usepackage{booktabs}
\usepackage[authoryear,round]{natbib}
\bibliographystyle{apalike}
\usepackage{bibentry}
\nobibliography{*}
\usepackage[colorlinks=true]{hyperref}
\usepackage{fancyref}

\begin{document}
%\maketitle

\begin{enumerate}
\item \textbf{Course description.} The goals of this course are to introduce the basic concepts of physics, to discover how science works as a process, and to learn more about the underlying processes that explain how the world works. The course can be thought of as a survey of different topics in physics and is intended to build on and extend students' knowledge from their middle school science classes. As the first course in the MBS upper school’s standard science curriculum, this course will also lay the foundation for future scientific work.

\item \textbf{Plan.} Units, each covering roughly a quarter, will include:
\begin{enumerate}
\item Electricity and magnetism
\item Waves, light and sound
\item Motion and forces
\item Work and energy
\end{enumerate}

\item \textbf{Evaluation.} 
\begin{enumerate}
\item Grades will be computed using the following breakdown:
\begin{table}[h]
\begin{center}
\begin{tabular}{ll}
\toprule
Quizzes & 25\% \\
Labs & 25\% \\
Quarterly tests & 20\% \\
Quarterly projects & 20\% \\
Diligence, curiosity, and growth & 10\% \\
\bottomrule
\end{tabular}
\end{center}
{\scriptsize
$^1$Homework problems will be assigned regularly, but assessed only for completion.\\
$^2$Quizzes and labs will be assigned (roughly) bi-weekly, meaning that one week you will have a quiz, then a lab the next week and so on.}
\end{table}

\item As grades are reported using the letter grade system, a holistic description in \fref{tab:grades}.
\begin{table}[h]
\caption{Letter grade criteria}
\label{tab:grades}
\begin{center}
\small
\begin{tabular}{lp{4in}}
\toprule
A & a complete mastery of all elements of curriculum; student consistently demonstrates independent thinking and a deep conceptual awareness of the breadth and depth of the field of study; student is capable of completely independent and self-guided study of the field; student is deeply motivated \\
B & student has mastered the essential elements of the curriculum; student occasionally exhibits independent thinking; student can occasionally handle unfamiliar material independently \\
C & limited mastery of course requirements; exhibits rote command of material but little conceptual understanding; foundational understanding is adequate but student is not (yet) developing conceptual connections or exhibiting an independent
understanding of the material \\
D & rarely applies any effort towards mastery of course requirements; infrequently demonstrates the most basic understanding\\
F & complete failure to meet even the most basic requirements of the course; total lack of effort \\
\bottomrule
\end{tabular}
\end{center}
\end{table}
\end{enumerate}

\item \textbf{Course materials.}
\begin{enumerate}
\item \textbf{Calculators.}  Please bring a scientific calculator \textbf{to every class} (any model will do). Graphing calculators are not necessary for this course, but the math department will likely require them at some stage, so consider buying one now (TI-84 or equivalent). \textbf{Please note that your phone is not acceptable as a calculator for tests/quizzes.}

\item \textbf{Textbook.} Textbooks can be quite expensive and so there will be no requirement to purchase one for this class. Instead, we will primarily use \bibentry{henderson2020physics}, available for free online. Students will be directed to the appropriate sections of the site as homework problems and readings are assigned.
\item Students may choose to take notes on paper or electronically. However, students are required to bring an electronic device (iPad or laptop) to all classes. All class notes written on the instructors iPad will be provided to the students via the resources folder on the class webpage.
\end{enumerate}

\item \textbf{Honors.} All Physics 9 students may elect to pursue an honors designation in the course. There will be a series of major assignments throughout the year that will be ``differentiated,'' meaning that they will have regular and honors options. The school policy is that in order to earn the honors designation a student must complete two-thirds of these assessments at the honors level earning an average grade of a B or higher. In this class, there will be eight honors assessments offered -- the four quarterly tests and the four quarterly projects -- meaning that you will need to complete six of these over the course of the year should you desire to achieve the honors designation. In order to assist students in deciding whether or not to pursue honors, at the start of the year the assigned homework problems will include some questions appropriate for honors. Sample honors test questions will also be provided as we approach the first test. For the projects, the project requirements and grading rubrics for both the regular and honors options will be made available when each project is assigned.

\item \textbf{Assistance.} This class will move quickly at times and can be challenging, so it is expected that most students will need to come for extra help periodically. The best time to be sure of finding me is during the collaborative periods built into the daily schedule. In addition, my free periods are 2, 3, 4, 7 and 8. Finally, please note that I coach MBS teams in both the fall and winter, so will occasionally leave early for games. It is always a good idea to confirm my availability via email ahead of time before coming to look for me.

\item \textbf{Instructor contact information.}
\begin{table}[h]
\hspace{3in}\begin{tabular}{l}
Dr. Dennis Evangelista \\
MSC xxx \\
\emph{devangelista@mbs.net} \\
(973) xxx-xxxx ext. xxx\\
\end{tabular}
\end{table}
\end{enumerate}

\nobibliography{physics9.bib}
\end{document}
