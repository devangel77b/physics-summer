\documentclass{tufte-handout}

\usepackage[handout]{physics_summer}

\title{Handout \#1: Mechanics, kinematics}
\author{D.~Evangelista and A.~Hahn}
\date{\printdate{6/14/2021}}

\begin{document}
\maketitle
\begin{enumerate}
\item Suggested reading is chapter 1 in \citeauthor{kleppner2014introduction}\cite{kleppner2014introduction}. 
\item Based on exam coverage, we will do mechanics, electricity \& magnetism, and waves. You have chosen to work on mechanics first.
\item These notes can be cloned from Github and compiled in \LaTeX\sidenote{The source files are at \url{https://github.com/devangel77b/physics-summer.git}}
\end{enumerate}

\section{Kinematic variables}
A large part of physics deals with describing motion and the \textbf{mechanics} of how bodies move. \textbf{Kinematics} is an umbrella term for mathematical descriptions of motion, and is where studies of mechanics typically begin. 

Imagine Stanley in a little cat-sized space suit floating in space, moving at constant speed\sidenote{For this discussion, we will limit Stanley to moving at a slow constant speed, say \SIrange{1}{2}{\meter\per\second}. If Stanley is moving close to the speed of light, the discussion we are following will break down and will need to be more complicated. We'll just keep it simple for now and assume we are slow enough that \textbf{Newtonian} mechanics are a good approximation. All models are wrong, some models are useful.}. It is useful to review equations that describe his \textbf{position}, \textbf{velocity}, and \textbf{acceleration}. 

\subsection{Position}
For starters, let's imagine Stanley starts at out $x=\SI{0}{\meter}$ and is moving at \SI{1}{\meter\per\second}. Stanley's position after \SI{1}{\second} would be \SI{1}{\meter}; after \SI{2}{\second}, \SI{2}{\meter}, and so on. If I asked you to graph it you would plot it as shown in \fref{fig:position}.
\begin{marginfigure}
\caption{Stanley's position}
\label{fig:position}
\end{marginfigure}

Physics likes to get things in equation form in order to develop compact mathematical descriptions that reduce nature into a few, minimal, experimentally supported laws. Let's write an equation for Stanley's \textbf{position as a function of time,} $x(t)$:
\begin{equation}
x(t) = \SI{1}{\meter\per\second} t,
\end{equation}
where $x$ is his position and $t$ is time\sidenote{For now we will assume time $t$ is an independent variable and distinct from position. If Stanley were traveling close to the speed of light, we would have to rethink this and include space and time in a single four-vector called space-time where they are not independent of one another.}. 

If instead, Stanley started at $x(t=0)=x_0=\SI{1}{\meter}$, the equation would be
\begin{equation}
x(t) = \SI{1}{\meter\per\second} t + \SI{1}{\meter};
\end{equation}
or if he started at $x_0=\SI{2}{\meter}$,
\begin{equation}
x(t) = \SI{1}{\meter\per\second} t + \SI{2}{\meter};
\end{equation}
and so on, as shown in \fref{fig:position2}.
\begin{marginfigure}
\caption{Stanley's position with different starting points (initial position $x_0$)}
\label{fig:position2}
\end{marginfigure} 

If Stanley was traveling at \SI{2}{\meter\per\second} and started at $x_0=\SI{2}{\meter}$,
\begin{equation}
x(t) = \SI{2}{\meter\per\second} t + \SI{2}{\meter},
\label{eq:position3}
\end{equation}
as shown in \fref{fig:position3}.
\begin{marginfigure}
\caption{Stanley's position when he is traveling at \SI{2}{\meter\per\second} and starts at $x_0=\SI{2}{\meter}$.}
\label{fig:position3}
\end{marginfigure} 

From algebra class you might recognize the graphs shown in figures~\ref{fig:position}-\ref{fig:position3} as the graph of a line whose equation has the form
\begin{equation}
y = mx + b,
\label{eq:algebraline}
\end{equation}
taking care to note that our independent variable is $t$ and our dependent variable is $x$. The slope $m$ appears to be the speed Stanley is traveling (e.g. \SI{1}{\meter\per\second}, \SI{2}{\meter\per\second}, etc...) while the $y$-intercept $b$ is Stanley's initial starting position $x_0 = x(t=0)$\sidenote{This notation quirk often confuses students who have only just had algebra and are still adjusting to using symbols that have different, discipline-specific standard uses. To deal with this, I suggest putting these quantities on flash cards and reviewing them until it feels comfortable to think of $t$ as ``$x$'' in \fref{eq:algebraline}, $x$ as ``$y$'' in \fref{eq:algebraline}, and so on...}. Considering this form, we see that \fref{eq:position3} is of the form
\begin{equation}
x(t) = V_x t + x_0 
\end{equation}
where $x$ is position and $t$ is time. $V_x$ is the slope (like $m$ in \fref{eq:algebraline}) and is the constant speed Stanley is moving at. $x_0$ is the $y$-intercept (like $b$ in \fref{eq:algebraline}) and is his starting position. 

\newthought{Take a moment} to convince yourself that we've adequately covered all possible cases of Stanley moving at ``slow\sidenote{``Slow'' means his speed is say less than $0.8c$, where $c=\SI{3e8}{\meter\per\second}$ is the speed of light.}'' constant velocity and starting anywhere he might like in a one-dimensional world. Stanley could also have speed \SI{0}{\meter\per\second}; what would that do? He could move backwards; what would that look like? 

\subsection{Velocity}
Let's go back to the case where Stanley's position in space is described by 
\begin{equation}
x(t) = 2 t + 2,
\end{equation}
as shown in \fref{fig:velocity1}.
\begin{marginfigure}
\caption{Stanley's position and velocity}
\label{fig:velocity1}
\end{marginfigure}

If I asked you to measure Stanley's speed or velocity\sidenote{Even though they are sometimes used interchangeably in everyday colloquial English, there's a big difference between speed and velocity, but we'll come to that after we discuss vectors.} you might see how far he has gone $\Delta x$ after some time $\Delta t$ and divide the two. For example, between $t=0$ and $t=1$,
\begin{align}
x(1) &= 2(1)+2 &= 4 \\
x(0) &= 2(0)+2 &= 2 \\
\Delta x &= 4 - 2 &= 2
\end{align}
Combining this with $\Delta t = 1-0 = 1$ gives $V_x=\frac{2}{1}=\SI{2}{\meter\per\second}$. An astute observer will notice that this is also the \textbf{slope}\sidenote{Slope is rise over run!} of the position vs time graph in \fref{fig:velocity1}. We'll come back to that.  For now just notice that \textbf{velocity is the (time) rate of change of position}, e.g.
\begin{equation}
\text{velocity} = \dfrac{\text{change in position}}{\text{change in time}} = \dfrac{\Delta x}{\Delta t}
\end{equation}
The units of $x$ are \si{\meter}, the units of $t$ are \si{\second}, so the units of velocity $v$ are \si{\meter\per\second}. Velocity tells \textbf{how fast Stanley is moving, and in what direction}. This latter bit will become more evident when we discuss vectors next time. 

\subsection{Acceleration}
Again let's return to the case where Stanley's position and velocity are described by
\begin{align}
x(t) &= 2 t + 2 \\
v_x(t) &= 2. 
\end{align}
as shown in \fref{fig:acceleration1}.
\begin{marginfigure}
\caption{Stanley's position, velocity, and acceleration}
\label{fig:acceleration1}
\end{marginfigure}
How fast is Stanley's velocity changing? We said Stanley is at constant speed, so the rate at which his velocity changes is \SI{0}{\meter\per\second\squared}; in other words, he is not accelerating. We can see this by taking the change in velocity divided by the change in time, say for the interval between $t=0$ and $t=1$:
\begin{align}
v_x(1) &= 2 \\
v_x(0) &= 2 \\
\Delta v_x &= 2 - 2 = 0
\end{align}
Combining this with $\Delta t = 1-0 = 1$ gives $A_x=\frac{0}{1}=\SI{0}{\meter\per\second}$. An astute observer will notice that this is also the \textbf{slope}\sidenote{Slope is rise over run!} of the velocity vs time graph in \fref{fig:acceleration1}. We'll come back to that.  \textbf{Acceleration is the (time) rate of change of velocity}, e.g.
\begin{equation}
\text{acceleration} = \dfrac{\text{change in velocity}}{\text{change in time}} = \dfrac{\Delta v}{\Delta t}
\end{equation}
The units of $v$ are \si{\meter\per\second}, the units of $t$ are \si{\second}, so the units of acceleration $a$ are \si{\meter\per\second\squared}. While velocity tells how fast Stanley is moving, and in what direction, \textbf{acceleration tells how fast his velocity is changing}. 

\newthought{Students sometimes get confused} about the difference between velocity and acceleration. For example, they think of a car and reason that a car goes fast (high speed) when you step on the gas (``accelerator'' pedal). What this misses, however, is that when you step on the gas pedal in a car it does not \emph{instantaneously} go fast... it accelerates and builds from a low speed to faster and faster speeds (more on this later), and that build up takes time. Velocity is like the speed. Acceleration is the \emph{rate} that velocity increases. It is possible to have high (positive) acceleration while still being at zero speed or even while heading backwards. In addition, it is possible to have a high speed (e.g. \SI{65}{mph}) and \emph{not} be accelerating\sidenote{...such as when you are driving on the NJ Turnpike at \emph{constant} speed, or if you are a space cat heading off towards the cosmos at constant speed...}. \textbf{Acceleration and velocity are different things, they have different units (\si{\meter\per\second\squared} vs \si{\meter\per\second}), best to make sure you get the two straight.} 



\section{Rates of change and derivatives; relationships between kinematics variables}
We said the velocity is the time rate of change of position:
\begin{equation}
\text{velocity} = \dfrac{\text{change in position}}{\text{change in time}} = \dfrac{\Delta x}{\Delta t}.
\label{eq:derivatives1}
\end{equation}
As a thought experiment, imagine how you might measure this in practice. You could measure Stanley's position at $t=\SI{0}{\second}$, then wait one second, and measure it again at $t=\SI{1}{\second}$. Would that work? Would your estimate be ``better'' if you waited for an hour? What if instead you only waited \SI{0.001}{\second} before measuring position again? 

Now comes a (tiny, painless, friendly)\sidenote{Calculus theme song here: \url{https://www.youtube.com/watch?v=ioE_O7Lm0I4}} piece of \textbf{calculus}\cite{greenspan1987calculus, kreysig2015advanced}. Let me imagine taking a ``really small\sidenote{The SAT word for this is \textbf{infinitesimal}, meaning really tiny, vanishingly so.}|'' $\Delta t$. Practically, this could be like \SI{0.001}{\second}, or some other small number. You may recall an idea of taking a \textbf{limit} as $\Delta t$ approaches 0 (gets smaller), which is mathematician-speak for get close to zero (get ``really small'') but do not actually reach zero. This last bit is just because we don't like to divide by zero. Rewriting \fref{eq:derivatives1} gives
\begin{equation}
v = \lim_{\Delta t\to 0} \dfrac{\Delta x}{\Delta t}, 
\label{eq:derivatives2}
\end{equation}
where the whole $\lim_{\Delta t\to 0}$ mumbo jumbo is read as ``the limit as $\Delta t$ approaches zero''. 

Calculus has a name for \fref{eq:derivatives2}, it's called the \textbf{derivative}:
\begin{equation}
v = \lim_{\Delta t\to 0} \dfrac{\Delta x}{\Delta t} = \dfrac{dx}{dt}.
\label{eq:derivatives3}
\end{equation}
In the calculus shorthand notation (from Leibniz) it is written as $\dfrac{dx}{dt}$, where little $d$ have replaced $\Delta$... so it shouldn't be too scary because all it means is it is the (local) slope at a point on the position $x$ vs time $t$ curve. You can think of it as giving the \textbf{slope of a line tangent to the curve at a particular instant in time $t$}. The derivative is tailor-made to help you find velocity when given position; and to find acceleration when given velocity. The relationships between these quantities are shown in \fref{fig:derivatives1}.
\begin{figure}
\caption{Relationships between position, velocity, and acceleration}
\label{fig:derivatives1}
\end{figure}
Mathematically, the derivative can be found by evaluating the limit... but if you have ever studied calculus, you probably vaguely remember there are rules\sidenote{Do the limit once to get the rule so you know where it comes from, then learn the rules so you can do this fast.} for taking the derivatives of certain functions (see \fref{tab:derivatives}). . 
\begin{margintable}
\caption{Some useful derivatives for simple kinematics}
\label{tab:derivatives}
\begin{center}
\small
\begin{tabular}{cc}
\toprule
$f(t)$ & $\int f(t) dt$ \\
\midrule
$0$ & $0$ \\
$1$ & $0$ \\
$a$ & $0^*$\\
$t$ & $1$ \\
$at$ & $a^\dag$\\
$t^2$ & $2$ t \\
$t^3$ & $3 t^2$ \\
$t^b$ & $b t^{b-1}$ \\
$at^b$ & $a b t^{b-1 \ddag}$\\
\bottomrule
\end{tabular}
\end{center}
\scriptsize
$^*$, derivative of constant is zero
$^\dag$, can pull out multiplicative constants
$^\ddag$, useful for polynomials  
\end{margintable}

Our whole Stanley moving in space at constant velocity then becomes pretty simple using the rules:\sidenote{Even though the 9th graders do not use calculus to solve these, it's actually much easier if you do allow yourself this little bit of calculus as you can get the solutions much quicker... by inspection even, if you remember how to take a derivative.} 
\begin{align}
\text{Given}\quad x(t) &= V t + x_0 \\
v(t) = \frac{d}{dt} x(t) &= V + 0 = V\quad\text{(take first derivative)} \\
a(t) = \frac{d}{dt} v(t) &= 0\quad\text{(take second derivative)}
\label{eq:derivatives4}
\end{align}

\newthought{If Stanley's position had a more complicated form}, such as, say $x(t)=\frac{1}{2}at^2 + v_0 t + x_0$, you could still do\sidenote{Try it!} the steps in \fref{eq:derivatives4}, \textbf{taking the first and second derivatives of position to get his velocity and acceleration, respectively}. This works for pretty much \textbf{any} differentiable\sidenote{Does \textbf{not} have to be smooth... Stanley could be stationary and then fire a thruster and would have a non-smooth velocity.} function $x(t)$, which physically means \textbf{Stanley is not moving at infinite speed or in two places at once}. We will allow ourselves these mathematical constraints because we said we were restricting our study to ``slow'' and also presumably non-quantum spacecats. 

Derivatives gave us a way to go from position to velocity to acceleration. \newthought{Is there a way to go from acceleration to velocity to position?}

\section{Areas and integrals; another set of relationships between kinematics variables}
Let us return to the case of motion at constant velocity: 
\begin{align}
x(t) &= V t + x_0 \\
v(t) &= V \\
a(t) &= 0.
\end{align}

For starters, can we get position from velocity? Sure, that's easy, because in math class we have drilled ourselves to know distance is speed times time:
\begin{align}
\text{distance} &= \text{speed} \times \text{time} \\
x &= V t
\label{eq:integrals1}
\end{align}

A graphical interpretation of \fref{eq:integrals1} is that position looks like the \textbf{area} of a rectangle whose height is $V$ and whose width is $t$. In other words, it appears to be the area under the curve of $v(t)$, less a constant term $x_0$. Similarly, the velocity looks like the area of a rectangle whose height is $0$, whose width is $t$, less a constant term $V$. 

In calculus, when we start dealing with areas under a curve, we are dealing with \textbf{integrals}\cite{greenspan1987calculus, kreysig2015advanced}. Along with derivatives, they are useful here because they give us a way to go from acceleration to velocity to position for any (integrable) case of acceleration. Since acceleration is tied to the force exerted on an object, this is pretty useful for finding equations of motion from forces acting on a body, which is a major task for mechanics within physics. For further details, refer to \cite{greenspan1987calculus}, but the general idera is to break a curve into many little rectangles of width $\Delta t$ and then take the limit as $\Delta t$ approaches zero. Where have we heard that idea before? We did something like that when taking derivatives, and there is a relationship between integrals and (anti)derivatives. 

Let's try the constant acceleration case in two ways: using areas (from geometry) as well as seeing if we can connect integrals or antiderivatives. For constant acceleration:
\begin{equation}
a(t) = A\quad\text{(constant)}.
\end{equation}
Let's imagine $v(t)$ as the area of a rectangle of height $A$ and width $t$:
\begin{equation}
v(t) = A t + C_1
\end{equation}
where $C_1$ is that constant we had to think about before. $x(t)$ ought to be the area of $v(t)$, which we can break up into a rectangle of constant height $C_1$, and a little triangle piece of height $A t$ and width $t$. The area of the rectangle is $C_1 t$, while the area of the triangle piece is $\frac{1}{2} b h$ or $\frac{1}{2} (t) (At)$:
\begin{equation}
x(t) = \frac{1}{2} A t^2 + C_1 t + C_2. 
\end{equation}
Is it right? Well taking the first derivative gives $v(t) = \frac{d}{dt}x(t) = A t + C_1$, and taking the second derivative gives $a(t) = \frac{d^2}{dt^2} x(t) = A$ which seems right. Also notice that the area under $a(t)$ seems to be a function $v(t)$ whose derivative is $a(t)$; and the area under $v(t)$ seems to be a function $x(t)$ whose derivative is $v(t)$. More generally,
\begin{align}
\text{If}\quad f(t) &= \frac{d}{dt} F(t) \\
f(t) &= \frac{dF}{dt} \\
f(t) dt &= dF \\
\int f(t) dt &= F(t) + C.
\end{align}
The integral $\int$ takes many, many little rectangles and sums them up (the sign is supposed to look like a stylized S representing sum). $C$ is the constant of integration.  Integrals are often derived via antiderivatives or other clever steps, but once you figure them out you can see and remember patterns, which are often provided in tables of integrals, for example, see \fref{tab:integrals}
\begin{margintable}
\caption{Some useful integrals for simple kinematics}
\label{tab:integrals}
\begin{center}
\small
\begin{tabular}{cc}
\toprule
$f(t)$ & $\int f(t) dt$ \\
\midrule
$0$ & $C$ \\
$1$ & $t + C$ \\
$a$ & $at + C$ \\
$t$ & $\frac{1}{2} t^2 + C$ \\
$at$ & $\frac{1}{2} at^2 + C$ \\
$t^2$ & $\frac{1}{3} t^3 + C$ \\
$a t^b$ & $\frac{a}{b+1} t^{b+1} + C$ \\
\bottomrule
\end{tabular}
\end{center}
\end{margintable}

\newthought{To take it back to physics}, \textbf{derivatives gave us a way to go from position to velocity to acceleration}:
\begin{align}
\text{(position)}\quad x(t) & \\
\text{(velocity)}\quad v(t) &= \frac{d}{dt} x(t)\quad\text{(first derivative)} \\
\text{(acceleration)}\quad a(t) &= \frac{d}{dt} v(t) = \frac{d^2}{dt^2} x(t)\quad\text{(second derivative)}.
\end{align}
\textbf{Integrals give us a way to go from acceleration to velocity to position}:
\begin{align}
\text{(acceleration)}\quad a(t) & \\
\text{(velocity)}\quad v(t) &= \int a(t) dt + C_1 \\
\text{(position)}\quad x(t) &= \int v(t) dt + C_2.
\end{align}




\section{Next time: vectors}
So far, we have considered only the case of one-dimensional (1D) motion. Stanley is traveling through space in a straight line and his position can be represented by a single coordinate $x$. Clearly we might wish to examine motion in two or three dimensions. To do so we will extend our notions of position, velocity, and acceleration using \textbf{vectors}, which will give us a shorthand way to handle two or three (or more) dimensions all at once. 

\bibliography{physics9}
\end{document}
